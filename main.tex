% ---------------------------------------------
% Autor:        Melanie Windrich, Timo Wilgen 
% Vorlage:      Daniel Fötsch, Sven Feja, Eckhard Anders
% Beschreibung: Hauptdatei der Diplomarbeit
% Dateiname:    main.tex 
% Projekt:      Diplomarbeit
% erstellt am:  31.10.2010
% geändert am:  06.12.2018
% SVN:		$Id$
% ---------------------------------------------
\documentclass[
	pdftex,			% Seitenangaben ins pdf schreiben
	11pt,			% Schriftgröße 12pt
	a4paper, 		% Layout für DIN A4
	twoside, 		% Layout für zweiseitigen Druck
	%fleqn,			% Formeln links eingerückt (default mittig)
	BCOR=2cm,		% Binderandkorrektur
	DIV=11,			% a little more room on page
	listof=totoc
]{scrbook} 

% Packages
\usepackage{scrhack}           % KomaScript compatibility hacks (e.g. for listing)
\usepackage[utf8]{inputenc}    % Paket für Eingabekodierung 

\usepackage[T1]{fontenc}       % T1-kodierte Schriften, korrekte Trennmuster fuer Worte mit Umlauten
%\usepackage{lmodern}		   % benutze Latin Modern als Standardschriftart
\usepackage{mathptmx}		   % benutze Times als Standardschriftart
\usepackage{helvet}		       % Serifenlose Schriftart
\usepackage{ifpdf}		       % Parameter zur Erkennung des Kompilierungstyps
\usepackage{listings}		   % Code einbinden
\usepackage{microtype} 		   % unterbindet full boxes durch bessere Anpassung der Schrift 

\usepackage{amssymb}		   % mathematische Symbole \mathbb{N} z.B.
\usepackage{nomencl}		   % Abkürzungsverzeichnis
\usepackage{xspace}		       % Leerzeichenproblem bei Commands lösen
\usepackage{setspace}		   % benötigt für Zeilenabstände
\usepackage{booktabs}		   % \toprule für Tabellen
\usepackage{enumitem}
\usepackage[hyphens]{url}
\usepackage[pdftex]{graphicx}  % Einbinden von Grafiken
\usepackage[pdftex]{color}     % Einbinden von Farben
\usepackage[pdftex]{xcolor}
\usepackage[pdftex]{hyperref}  % Einbinden von Verlinkung
\usepackage{todonotes}		   % \todo{} befehl [disable] zum deaktivieren
\usepackage{pdfpages} 	       % Einbinden von pdf
\usepackage{cite}		       % Für bibtex Parsing unter Windows
\usepackage{blindtext}
\usepackage{amsmath}		   % AMS Mathe und Symbolpakete für mehr Möglichkeiten
\usepackage{amsfonts}
\usepackage{amssymb}
\usepackage{verbatim}          % \begin{comment}, \end{comment}


\usepackage{multirow}		   % Zeilen in Tabellen zusammenfassen (http://tug.ctan.org/tex-archive/macros/latex/required/graphics/grfguide.pdf)
\usepackage{hhline}		       % Horizontale Linien in Tabellen feiner gestalten
\usepackage{longtable}

% Unterabbildungen können mit dem subfig-Paket realisiert werden.
% (z.B. Abb. 1a) (http://tug.ctan.org/tex-archive/macros/latex/contrib/subfig/subfig.pdf)
\usepackage{subfig}

% Großartige Möglichkeiten um Grafiken direkt in LaTeX zu erzeugen bietet Tikz 
% (http://mirror.ctan.org/graphics/pgf/base/doc/generic/pgf/pgfmanual.pdf und http://www.statistiker-wg.de/pgf/tutorials.htm)
\usepackage{tikz}		% 
\usetikzlibrary[shapes.misc,shapes.geometric,arrows,decorations.pathmorphing,backgrounds,fit,positioning]

% \setlength{\headheight}{10mm}
% \setlength{\headsep}{8mm}  

% \setcounter{secnumdepth}{3}           % Globale Tiefe im Dokument
% \setcounter{tocdepth}{3}              % Tiefe der Numerierung im Inhaltsverzeichnis
\renewcommand{\topfraction}{0.99}
\renewcommand{\textfraction}{0.01}
% Die vorangegangenen Befehle sind noetig, damit grosse Objekte nicht
% erst auf der letzten Seite erscheinen... (siehe Kopka-Buch S. 170)

\setcounter{topnumber}{9}
\setcounter{totalnumber}{9}
% Damit mehr Tabellen/Abbildungen auf eine Seite passen. (S. 170)

% Hurenkinder und Schusterjungen verbieten
\clubpenalty = 10000
\widowpenalty = 10000
\displaywidowpenalty = 10000

% Definition von Farben
\definecolor{darkred}{rgb}{0.5,0,0}
\definecolor{darkgreen}{rgb}{0,0.3,0}
\definecolor{darkblue}{rgb}{0,0,0.5}
\definecolor{black}{rgb}{0.2,0.2,0.2}
\definecolor{darkbrown}{rgb}{0.28,0.07,0.07}

% Einstellungs für Abkürzungsverzeichnis
\let\abk\nomenclature					% Befehl umbenennen in abk
\setlength{\nomlabelwidth}{.20\hsize}			% Punkte zw. Abkürzung und Erklärung
\renewcommand{\nomlabel}[1]{#1 \dotfill}
\renewcommand*\contentsname{Content}
\setlength{\nomitemsep}{-\parsep}			% Zeilenabstände verkleinern
\makenomenclature

% Zeilenabstand für Inhalte
\newcommand{\contentspacing}{\setstretch{1.1}}
% \newcommand{\contentspacing}{\singlespacing}

% settings
\newcommand{\bitAuthor}{Anton Wagner}
\newcommand{\bitTitle}{Predicting Gaussians: 3D Scene Reconstruction using Transformers}
\newcommand{\bitType}{Bachelorthesis}
\newcommand{\bitZweitbetreuer}{Second Assessor:}
% 
% Es ist EIN Zweitbetreuer anzugeben!
% bspw.
% Betreuer in der Firma
% M.Sc. Melanie Windrich
% ...
\newcommand{\bitDeadline}{März 2024}
\newcommand{\bitKeywords}{Some key words}

\hypersetup{
	colorlinks=false,       % Anzeigen der Links in Farbe
	pdftitle={\bitTitle},
	pdfauthor={\bitAuthor},
	pdfsubject={\bitType},
	pdfkeywords={\bitKeywords}, 
	plainpages=false,	    % Notiz, dass Duplikate ignoriert werden
	hypertexnames=false,    % Links im Index nicht aktiviert
%	bookmarks=true,        	% Anzeigen oder Verstecken der Bookmarks 
	pdffitwindow=true,      % Setzen von initialen Parametern fr das Anzeigen des PDF-Dokumentes
	linkcolor=black,        % Colorierung von internen Links (Abschnitten, Seiten, etc.),
	citecolor=darkblue,     % Colorierung der Links auf Referenzen
	urlcolor=darkgreen,     % Colorierung der Links auf WWW-Seiten, z.B. \href{http://www.ctan.org}{CTAN}
	menucolor=darkbrown,    % Colorierung der Links im Men
	filecolor=magenta       % Colorierung der Links auf Dateien, z.B. \href{manual.pdf}{here}
}

\pdfcompresslevel=9
\DeclareGraphicsExtensions{.pdf, .png, .jpg, .mps}

% --------------------------- Definitionen -------------------------------
\renewcommand{\lstlistingname}{Listing}
\renewcommand{\lstlistlistingname}{Quellcodeverzeichnis}

\lstset{%
	language=java,
	commentstyle=\color{darkgreen},
	stringstyle=\color{darkred},
	basicstyle=\footnotesize\ttfamily,
	tabsize=2,
	numbers=left,
	numberstyle=\tiny,
	breakautoindent  = true,
	breakindent      = 2em,
	breaklines       = true,
% 	prebreak         = \raisebox{-.8ex}[0ex][0ex]{\Righttorque},
	showspaces=false, 	% Keine Leerzeichensymbole
	showtabs=false, 	% Keine Tabsymbole
	showstringspaces=false,	% Leerzeichen in Strings
	frame=lines,
	captionpos=b
}

% ---------------------------------------------
% Autor:        Eckhard Anders
% Vorlage:      Daniel Fötsch, Sven Feja
% Beschreibung: Allgemeine Befehle
% Dateiname:    commands.tex 
% Projekt:      Diplomarbeit
% erstellt am:  31.10.2010
% geändert am:  $Date$
% SVN:			$Id$
% ---------------------------------------------


% Code
\newcommand{\code}[1]{\textsf{#1}}	% Code in Text
\newcommand{\pkg}[1]{\textsf{#1}}	% Paketnamen
\newcommand{\class}[1]{\textsf{#1}} 	% Klassennamen
\newcommand{\idx}[1]{_\text{#1}} 	% schreibt in Formeln einen nicht kursiven Index

\newcommand{\quellcode}{\lstlistingname\xspace}
\newcommand{\listing}{Listing\xspace} 

\newcommand{\macos}{Mac OS\xspace}
% ...

% Standards
\newcommand{\zB}{z.\,B.\ }
\newcommand{\idR}{i.\,d.\,R.\ }
\newcommand{\bzw}{bzw.\ }
\newcommand{\etc}{etc.\ }
\newcommand{\iA}{i.\,A.\ }
\newcommand{\uU}{u.\,U.\ }
\newcommand{\ua}{u.\,a.\ }
\newcommand{\usw}{usw.\ }
\renewcommand{\dh}{d.\,h.\ }

     		% Nutze eigene Commands


% Erstes Kapitel
\abk{z.B.}{zum Beispiel}

\bibliographystyle{alphadin}	% Literaturstil setzen

% ---------------------------Dokument--------------------------
\begin{document}
\pagenumbering{roman} 		% römische Seitenzahlen mittig
\pagestyle{plain}		% Standard Seitenstil 

% ---------------------- Dokument jetzt wirklich --------------------------
% Titelblatt
% ---------------------------------------------
% Autor:        Melanie Windrich, Timo Wilgen 
% Vorlage:      Daniel Fötsch, Sven Feja, Eckhard Anders
% Beschreibung: Hauptdatei der Diplomarbeit
% Dateiname:    main.tex 
% Projekt:      Diplomarbeit
% erstellt am:  31.10.2010
% geändert am:  06.12.2018
% SVN:		$Id$
% ---------------------------------------------

\begin{titlepage}

  \begin{center}    
      \LARGE Christian-Albrechts-Universität zu Kiel\\
      \vspace{0.2cm}
      \large Department of Computer Science\\ 
    \begin{LARGE}
    \vspace*{2.5cm}

    \vspace*{2.5cm}
    \doublespacing
    \bitType\\
    \vspace*{0.3cm}
    {\sffamily\bfseries\bitTitle}
    \singlespacing
    \end{LARGE}
    \vspace{-0.25cm}
    {\LARGE \bitAuthor}\\[1.5cm]
    \vfill
    {First Assessor:}\\[0.1cm] 
    {Prof. Dr. Sören Pirk}\\[0.1cm]
    {\bitZweitbetreuer}\\[0.1cm]
    {Ma.Sc. Helge Wrede}\\[0.5cm] 
    {\small Christian-Albrechts-Universität zu Kiel\\
			Department of Computer Science\\
			Visual Computing and Artificial Intelligence\\
		  	Neufeldtstraße. 6, 24118 Kiel
	}\\[0.5cm]
			
	{\bitDeadline}
  \end{center}
\end{titlepage} 
			% Deckblatt
\clearpage{\pagestyle{empty}\cleardoublepage}
 
% ---------------------------Abstract-------------------------------------
\contentspacing
% ---------------------------------------------
% Autor:        Eckhard Anders
% Vorlage:      Daniel Fötsch, Sven Feja
% Beschreibung: Zusammenfassung
% Dateiname:    0_summary.tex 
% Projekt:      Diplomarbeit
% erstellt am:  31.10.2010
% geändert am:  $Date$
% SVN:			$Id$
% ---------------------------------------------

\chapter*{Kurzfassung}
\label{sec:kurzfassung}
\addcontentsline{toc}{chapter}{Kurzfassung}

Eine Kurzfassung der Arbeit.

Länge maximal eine Seite.

Eventuell folgt eine Auflistung von Schlüsselworten, die den Inhalt der Arbeit beschreiben: 
\begin{itemize}
 \item XML
 \item Modelltransformation
 \item Validierung
\end{itemize}		% Abstract deutsch
\clearpage{\pagestyle{empty}\cleardoublepage}

% ---------------------------Inhaltsverzeichnisse--------------------------
\singlespacing
\tableofcontents		% Table of Contents
% \addcontentsline{toc}{chapter}{Table of Contents}
\addtocontents{toc}{\protect\addcontentsline{toc}{chapter}{Table of Contents}}
\contentspacing

% ---------------------------Hauptteil--------------------------------------
\mainmatter			% Beginn des Hauptteils
\pagenumbering{arabic}		% arabische Seitenzahlen mittig 
\setcounter{page}{1}		% Zurücksetzen des Seitenzählers

\chapter{Related Works}

This Chapter will discuss the origins and inspirations for different components of the final network, as well as other works that use 3D Gaussian Splatting. Relevant sources will be:
\cite{kerbl3Dgaussians}
\cite{zou2023triplane}
\cite{yu2021pointbert}
\cite{yu2021pointr}


I will also mention the similarity to vision transformers such as:
\cite{devlin2019bert}
\cite{yu2021diverse}
\cite{Miao2024}


\chapter{Introduction}
This chapter will explain 3D Gaussians and their ability for novel HD view synthesis at high framerates \cite{kerbl3Dgaussians}
It will also give a preliminary explanation as to why i chose a transformer architecture. \cite{vaswani2023attention}

\section{Motivation}
[An Image showing a hole in an otherwise good Scene]\\
This section will highlight the large interest being shown towards gaussians and will explain how gaussian scenes end up with holes that need to be fixed.

\section{Goals}
This section describes my goal of trying to have an existing gaussian scene be understood by a transformer and then continued. Elaborating on the idea of getting a "good" latent space representation of gaussians.

\section{Challenges}
This section will describe the non structural nature of gaussians and the difficulties of regressing them directly \cite{zou2023triplane}.
It will also show a diagram of the attention layer and it's quadratic memory/computational complexity, describing the need to reduce the sequence length.

\chapter{Network Architecture}
This chapter will start of with a diagram and description of the rough structure of the entire network.

\section{Visual Embedding}
This section will describe the Visual Embedding Net with a more closed up diagram. Both the novel encoder and the decoder (same as used in \cite{zou2023triplane}) will be elaborated on.
There will also be some pictures showing the capabilities of the Visual to encode and decode gaussians while retaining good image quality.

\section{DVAE}
This chapter will introduce the DVAE \cite{rolfe2017discrete} as a way comprehend a local pointcloud and convert it to a token and vice-versa.
\subsection{Sampling/Grouping}
This section will elaborate on the underwhelming performance of the commonly used Furthest-Point + KNN Sampling and will compare and contrast it with Random-Point + KNN Sampling.
\subsection{DGCNN}
This section will explain the DGCNN \cite{wang2019dynamic} and how it enables the DVAE to understand local geometries
\subsection{Discretization}
This section will introduce the vocabulary/codebook and how Gumbel-Softmax \cite{jang2017categorical} is used to discretize the resulting logits before sampling from the vocabulary.
\section{Transformer}
This section will talk about the transformer as a whole.
\subsection{Encoder}
This subsection will describe the self-attention mechanism and the positional encoding used in the encoder-blocks of the transformer.
\subsection{Query Generator}
This subsection will describe the Query generator as a way to turn the memory tokens generated by the encoder into positional information of where the missing content is located.
\subsection{Decoder}
This subsection will describe the cross-attention mechanism used to combine the memory and missing content tokens into useful output tokens.
\chapter{Training}
This chapter will explain the training regime used for the transformer. Describing both methods: 1. Training everything together vs 2. Training every component seperately 
\chapter{Results}
This chapter will look at some results, evaluating them both qualitatively and quantitavely. 
\section{Overall Evaluation}
This subsection will focus on the final results obtained.
\section{Vocabulary Analysis}
This subsection will analyze the different tokens learned by the DVAE.

\chapter{Limitations}
This chapter will highlight the limitations currently present in both the architecture and also results obtained

\chapter{Further Work}
\section{Improvements}
This chapter will somewhat speculate on ways that the architecture could be improved and goals i have in mind for future work on this particular problem
\section{Prospects}
This chapter will talk about the things that would become possible if I can achieve a good latent space representation of Gaussian Scenes (such as style-transfer, etc).






% \subsubsection{Vierte Gliederungsebene}

% Drei Gliederungsebenen sollten für kleine Artikel und Ausarbeitungen genügen. 
% Anderenfalls sollte die Strukturierung nocheinmal überdacht werden.

% \paragraph{Paragraph ist die unterste Ebene} \blindtext

% \section{Tabellen einbinden}

% \begin{table}
% \centering
% \begin{tabular}{|c||lr|}\hline
% 1.1 & 1.2 & 1.3 \\ \hline
% 2.1 & 2.2 & 2.3 \\ \hline \hline
% 3.1 & 3.2 & 3.3 \\ \hline
% \end{tabular}
% \caption{Eine simple Tabelle}\label{tab:simple}
% \end{table}


% Tabellen sollten in einer Table-Umgebung eingefügt und mit einer Caption und einem Label versehen werden.
% Ein einfaches Beispiel zeigt Tabelle \ref{tab:simple}.

% Leider sind Tabellen eines der schwierigeren Kapitel in \LaTeX,
% wenn beispielsweise Zellen zusammengefasst werden.
% Tabelle \ref{tab::komplex} zeigt eine etwas aufwendigere Tabelle.


% \begin{table}
% \centering
% \begin{tabular}{c|c|c|c|c|}	
% 	\multicolumn{2}{c}{}
% 	& \multicolumn{3}{c}{\begin{scriptsize}\textbf{GdO-Typ}\end{scriptsize}}
% 	\\ \hhline{~~---}
% 	\multicolumn{2}{c|}{}
% 	& \textbf{reellwertig}
% 	& \textbf{ganzzahlig}
% 	& \textbf{symbolisch}
% 	\\ \hhline{~|----}
% 	\multirow{4}[3]{*}{\rotatebox{90}{\begin{scriptsize}\textbf{EA-Typ}\end{scriptsize}}}
% 	& \textbf{reellwertig}
% 	& direkt
% 	& Rundung
% 	& ---
% 	\\ \hhline{~|----}
% 	& \textbf{ganzzahlig}
% 	& ---
% 	& direkt
% 	& Index
% 	\\ \hhline{~|----}
% 	& \textbf{Zeichenkette}
% 	& Interpretation
% 	& Interpretation
% 	& direkt
% 	\\ \hhline{~|----}
% \end{tabular}
% \caption{Beispiel für eine komplexere Tabelle}\label{tab::komplex}
% \end{table}

% \section{Bilder/Grafiken einbinden}

% Am besten werden Vektorgrafiken verwendet.
% Diese liegen im Idealfall als PDF vor.
% Aber auch EPS kann z.B. sehr einfach konvertiert werden.

% PDF-Grafiken können unter anderem mit Inkscape \cite{inkscape}, OpenOffice Draw \cite{oodraw} kostenlos bearbeitet und erstellt werden.
% Pixelgrafiken sollten unbedingt vermieden werden. 
% Ihre Auflösung sollte mind. 300dpi betragen.

% \subsection{Einfache Abbildungen}

% \begin{figure}
% \centering % zentriert alles in der Figure
% \includegraphics[width=0.6\linewidth]{Images/Chapter/bspgrafik1} % externe Grafik laden
% \caption{Dieses Diagramm ist ein Beispiel für eine einfache Abbildung.}\label{fig:testabb}
% \end{figure}


% Eine Abbildung sollte sich immer in einer Figure-Umgebung befinden.
% In dieser kann sie mit einer Caption beschrieben werden
% und sie kann über ein Label gekennzeichnet werden (vgl. Abbildung \ref{fig:testabb}).

% Dabei muss es keine Grafik sein, die in eine Figure-Umgebung geladen wird.
% Es kann dort ganz normal \LaTeX geschrieben werden,
% wie Abbildung \ref{fig:testabb2} zeigt.

% \begin{figure}
% \centering
% \fbox{\parbox{0.9\linewidth}{\centering In einer Figure muss keine Grafik stehen... Eine Abbildung kann im Prinzip alles sein.}}
% \caption{Dies ist eine Bildunterschrift} \label{fig:testabb2}
% \end{figure}

% \subsection{Unterabbildungen}

% \begin{figure}%
%   \centering
%    \subfloat[Die erste Unterabbildung]{\label{fig:subfig1}%
%        \includegraphics[width=0.48\linewidth]{Images/Chapter/bspgrafik1}
%    }\hfill
%    \subfloat[Die zweite Unterabbildung]{\label{fig:subfig2}%
%        \includegraphics[width=0.48\linewidth]{Images/Chapter/bspgrafik2}
%    }
%    \caption{Beispiel für Unterabbildungen}
%    \label{fig:subfigexample}
% \end{figure}


% In manchen Fällen ist es sinnvoll eine Abbildung in Unterabbildungen zu teilen.
% In Abbildung \ref{fig:subfigexample} wird dies gezeigt.
% Die Abbildungen \ref{fig:subfig1} und \ref{fig:subfig2} sind Unterabbildungen.

% \section{Formeln}

% Für seinen Formelsatz ist \LaTeX besonders bekannt, weshalb es hier nicht an einem kleinen Beispiel (vgl. Formel \ref{eq::gewichtsumme}) fehlen soll.

% \begin{equation}
% f\idx{sim}(g,U)=\sum\limits^{n\idx{M}}_{\mu=1} w_\mu\cdot c_i( M_\mu(g,U)) \label{eq::gewichtsumme}
% \end{equation}

% \section{Quelltexte einbinden}

% \begin{figure}[t]
% \begin{lstlisting}[language=java, caption={Beispiel für ein Listing},  label=lst:bsplst]
% public class RadioFitness extends BasicFitness {
%   ...
%   SimResultReader result = new SimResultReader();
%   ...
%   protected void readResult(Scenario s) throws FitnessException {
%     result.readFrom(s.getExecEnv().getExecutionDir());
%   }

%   protected double getRawMetric(String name) throws FitnessException {
%     if(name.equals("delivery rate")
%       return result.getNumSendData()/result.getNumReceivedData();
%     else if(name.equals("latency"))
%       return results.getLatency();
%     else ...
%   } // comment
%   ...
% }
% \end{lstlisting}
% \end{figure}


% Ein Beispiel für ein Java-Listing zeigt Listing \ref{lst:bsplst}.


% \section{Zitieren von Quellen}

% Aussagen wollen gut belegt sein.
% Hier sind willkürlich \cite{FejF08} beispielhafte Zitierungen \cite{Deming1986} angegeben,
% die keinen inhaltlichen Bezug zu diesem Text aufweisen.
% Vielmehr geht es darum beispielhaft zu zitieren.
% Es können auch mehrere Quellen angegeben werden \cite{biturl,Rost2009} (auch hier wieder ohne inhaltlichen Bezug).

% Die Literaturangaben werden in einer Datenbank verwaltet,
% die in einer .bib-Datei gespeichert wird.
% In dieser kann auch nicht zitierte Literatur stehen.
% Eine gute Software zur Bearbeitung dieser Datenbank ist JabRef \cite{jabref}.		% Kapitel 1 / n

% ---------------------------Anhang-----------------------------------------
\appendix
\chapter{Quelltexte}

\section{ Beispiel Java Klasse}
\label{app:java}

\lstinputlisting[language=Java,
caption={Draw2D Beispielklasse}]{Appendix/Files/RectangleExample.java}

		% Beispielklasse

% ---------------------------Verzeichnisse----------------------------------
\singlespacing 
\bibliography{literatur}	% Literaturverzeichnis
\addcontentsline{toc}{chapter}{\refname}
\listoffigures			% Abbildungsverzeichnis
\listoftables			% Tabellenverzeichnis
\lstlistoflistings		% Codeverzeichnis
\printnomenclature		% Abkürzungsverzeichnis

% ---------------------------Erklärung---------------------------------------
% ---------------------------------------------
% Autor:        Melanie Windrich, Timo Wilgen 
% Vorlage:      Daniel Fötsch, Sven Feja, Eckhard Anders
% Beschreibung: Hauptdatei der Diplomarbeit
% Dateiname:    main.tex 
% Projekt:      Diplomarbeit
% erstellt am:  31.10.2010
% geändert am:  03.04.2017
% SVN:		$Id$
% ---------------------------------------------

\contentspacing

\chapter*{Erklärung}
\label{sec:Erklaerung}
\addcontentsline{toc}{chapter}{Erklärung}

Hiermit erkläre ich,
dass ich die vorliegende Arbeit selbständig und ohne fremde Hilfe angefertigt und keine anderen als die angegebenen Quellen und Hilfsmittel verwendet habe.

% Optionaler Einschub falls der Arbeit die elektronische Fassung beigelegt wird.
\vspace{0.5cm}
\noindent Die eingereichte schriftliche Fassung der Arbeit entspricht der auf dem elektronischen Speichermedium.

\vspace{0.5cm}
\noindent Weiterhin versichere ich, dass diese Arbeit noch nicht als Abschlussarbeit an anderer Stelle vorgelegen hat.




\vspace{30mm}
Kiel, den \today
\vspace{5mm}

\raggedleft \bitAuthor

\end{document}